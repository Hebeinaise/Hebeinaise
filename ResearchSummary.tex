\documentclass[psamsfonts,reqno]{amsart}
\usepackage{geometry}                % See geometry.pdf to learn the layout options. There are lots.
\geometry{letterpaper}                   % ... or a4paper or a5paper or ... 
%\geometry{landscape}                % Activate for for rotated page geometry
\usepackage[parfill]{parskip}
\usepackage{graphicx}
\usepackage{amssymb}
\usepackage{epstopdf}
\usepackage{amsthm}

\usepackage{amsmath}
\usepackage{mathrsfs}
\usepackage{url}
\usepackage{verbatim}
\usepackage{multicol}
\input xy
\xyoption{all}
\usepackage[usenames,dvipsnames]{color}

\newtheorem{thm}{Theorem}
\newtheorem{prop}[thm]{Proposition}
\newtheorem{lem}[thm]{Lemma}
\newtheorem{col}[thm]{Corollary}
\newtheorem{claim}[thm]{Claim}

\theoremstyle{definition}
\newtheorem{defn}[thm]{Definition}
\newtheorem{alg}[thm]{Algorithm}
\newtheorem*{rmk}{Remark}
\newtheorem{tnk}[thm]{Technique}
\newtheorem*{tpc}{Topic}

\newcommand{\cyan}{\color{Cyan}}
\newcommand{\green}{\color{Green}}
\newcommand{\red}{\color{RedOrange}}
\newcommand{\mr}{\mathrm}
\newcommand{\mb}{\mathbb}
\newcommand{\mc}{\mathcal}

\title{Research Summary}
\author{Chiao-Yu Yang}
\date{}
\begin{document}
\maketitle
In the following, I will describe my research experiences I have had since my freshman year in chronological order. In the first two years, I mostly worked in discrete mathematics projects. In the summer after my sophomore year, I worked in a data mining project at UCLA, and I also started the study of numerical PDEs. In my junior year, I furthered my study of numerical methods for differential equations and also started to explore microscopic modeling in computational fluid dynamics. During the summer after my junior year, I worked on implementing a particle model under the supervision of Prof. Aleksandar Donev at Courant Institute. Currently, I am working with Prof. Donev on numerical tests of the particle model we developed.

\section*{Project 1: Parking functions, Spanning trees}

In the summer after my freshman year, I participated in a research project with Prof. David Perkinson and a fellow student Kuai Yu. Our project was about an open problem in combinatorics. In his book {\textit{Hyperplane Arrangements}}, Prof. Richard Stanley at MIT proposed the problem of finding a natural bijection between the set of parking functions and the set of spanning trees with specific inversions, a problem that had been unsolved for a few years. The technical difficulties are twofold. First, theoretically there are too many ways to construct the bijections and it takes a lot of time to test if a bijection is natural for different choices of graphs. Second, given that the cardinalities of both sets are equal, a bijection always exists but most of the bijections are not systematic as they cannot be reproduced for different graphs and they do not reveal any useful relationship between parking functions and spanning trees. We noticed that the bijection may exist for a particular group of graphs, the threshold graphs. However, after writing computer codes and running the code on a large set of graphs, I observed that for threshold graphs, there were a few exceptions that broke the bijection. Later, we tried to use ideas inspired by work of I. Gessel and J. Beissinger, but these all yielded messy results. However, after considering the problem in a more general way and viewing parking functions in a general way, we found that a pattern actually existed, for any connected graph. Then, a seemingly unrelated article by D. Dhar reminded me of some special structure of the parking functions and I came up with a bijection inspired by Dhar's paper. After our group meeting, we managed to prove that this bijection is the natural answer to the question. In addition, we found an algorithm much more efficient than the traditional one in computing the bijection. The result was accepted for publication in  {\em{Combinatorica}}.

\section*{Project 2: Sandpile, Plane duality}

At a workshop organized by the American Institute of Mathematics, M. Baker proposed a conjecture of whether rotor-routing action of the sandpile group of a planar graph $G$ on its spanning trees is compatible under planar duality with that of the sandpile group of its planar dual $G^*$ on its spanning trees. Prof. David Perkinson asked me to study the conjecture. Rotor-routing is defined for a directed graph with a sink. We fix a cyclic ordering on the outgoing edges for each vertex except the sink. Then, we start with a rotor for each vertex, i.e., we fix an outgoing edge. Now, a chip performs a walk, which we refer to as the rotor-routing process. Specifically, a chip starts at a non-sink vertex, and goes along the edge to the next vertex given by the rotor, then the rotor is changed according to the cyclic ordering. Once the chip reaches the sink, the rotor routing process ends. It was observed that for a planar graph, if rotor routing on a configuration $C_1$ resulted in $C_2$, then rotor routing on $C_1^*$, the dual of $C_1$, should result in $C_2^*$, the dual of $C_2$. We proved the result by using graph theoretic techniques. Specifically, we introduced the concept of angle between spanning trees and proved that the angle between two trees equals the angles between the duals of the two trees. Later, we communicated with several other participants of the workshop who showed interest in the problem. Together, we summarized the results and submitted them to {\textit{SIAM Journal of Discrete Mathematics}}. It was published in Volume 29 of year 2015.

\section*{Project 3: Realization of Generalized Joint Degree Matrix}

I went to study abroad in Budapest in my sophomore year and participated in a research group there, working with Prof. Istvan Miklos, in finding realizations of generalized JDM (joint degree matrix). A JDM is a matrix $J$ with dimension $\Delta \times \Delta$, where $\Delta$ is the maximum vertex degree and $J_{ij}$ is the number of edges between vertices with degree $i$ and vertices of degree $j$. Earlier, it had been proved that all realizations of a JDM, i.e., all graphs that satisfy the JDM, with fixed vertex set are connected via restricted swaps. Our project is focused on finding realizations of generalized joint degree matrix, where inside each vertex class the degrees may be different. We also explored partial JDM, similar to the conventional JDM except that it has some parts missing. During the semester, we tried edge-counting gadget, Havel-Hakimi type of algorithms, greedy algorithm, among others. We showed realization of partial JDM is in $P$ and realization of generalized JDM where one of the vertex sets is regular is also in $P$.

\section*{Project 4: Integrating Exterior Data to An Indexed Data Base}

I went to UCLA to participate in the RIPS (Research in Industrial Projects for Students) program in the summer of 2014 and worked on the team of the Shoah Foundation. The Shoah Foundation is a non-profit organization at USC that records testimonies in video format of survivors and other witnesses of the Holocaust and other massacres. They wanted to improve their website by providing the users with additional information relevant to the videos. We started with using tools from data mining such as term weighting techniques, query expansions, latent semantic indexing, among others. After doing tests on the database provided by Shoah Foundation, we decided to use Wikipedia as a test data base for the retrieval of information. In the second half of the program, we gathered data, used statistical methods to determine the efficiency of our algorithm, and used statistical learning techniques to improve the results. The final result was presented to the Shoah Foundation in a report.

\section*{Project 5: Weighted Essentially Non-Oscillation Scheme and Its Application}

During my time at UCLA, I talked with Prof. Li Wang and expressed my interest in numerical analysis. She gave me a list of reference books to read and asked me to implement the 5th order WENO (weighted essentially non-oscillatory) scheme by myself. I studied a list of papers on construction of the WENO scheme, including the original ENO paper by Harten, Engquist, Osher and Chakravarthy, the first WENO paper by Liu, Osher, and Chan, and several papers by CW Shu. After that I also studied application of WENO on semiconductor Boltzmann equations. Later, because of the time constraint, I stopped working on this project but plan to resume in January 2016.

\section*{Project 6: FEM Scheme for Parabolic $p$-Heat Equation}

I went to Moscow to study abroad in spring of 2015. While taking the graduate course in PDEs, I also worked with the instructor, Prof. Maxim Romanov, on a project investigating the parabolic $p$-heat equation. I studied theoretical aspects of this equation and in particular studied existence and uniqueness of the solution. After that we studied numerical schemes, especially finite element scheme and Galerkin method.

\section*{Project 7: Isotropic Particle Model for Reactive Systems}

After my junior year, I went to the Courant Institute at NYU to work under the supervision of Prof. Aleksandar Donev on developing a particle model that describes reaction-diffusion processes. Deterministic differential equations cannot provide an accurate answer to the evolution of certain systems. For example, combustion in a low-temperature system can lead to a state where thermal fluctuation is much stronger and thus causes macroscopic behavior to deviate from that given by deterministic differential equations. There was a particle method based on reaction-diffusion master equations. However, this method introduces artifacts and makes the system not Galilean invariant. To solve this problem, we used an isotropic method and allowed bimolecular reactions between particles in neighboring cells. To do so required a different time-scheduling scheme and reaction criteria. By now we have finished most of the model. I am also writing my senior thesis based on this work and also working with Prof. Donev on numerical tests and optimizations. After finishing the numerical test, we will write up the result and submit to the Journal of Chemical Physics.

 

\end{document}